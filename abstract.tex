\section{Abstract}
\label{abstract}

Beginning in October 2017, victims of sexual assault or harassment began using the hashtag “\#MeToo” to illustrate the magnitude of sexual assault victims. Dr. Naeemul Hassan has collected approximately half a million tweets that contain the hashtag \#MeToo and completed an initial analysis regarding various aspects of this data. However, Tweets that contain \#MeToo can either be from victims communicating a personal experience, people who are against the movement, or people who are in support of the movement but not victims themselves. Without knowing the semantics of the tweet itself, the existing analysis is not as useful as it could be if the tweets had been categorized (specifically analyzing the tweets of victims’ experiences to find trends). This project is to develop a program that can process these tweets and categorize them into one of the three categories described above (and further categorize the type of sexual harassment if possible), which would not only allow for more useful analysis of the data but could potentially be used in real-time to identify victims.

The “minimum viable product” version of this project is a program that can reliably and accurately categorize a tweet that contains \#MeToo as being either a personal experience, in support of the movement, or against it. This project will primarily require knowledge natural language processing (NLP) techniques and tools (CoreNLP, NLTK, spacy, etc.). Additionally, it will require knowledge of machine learning algorithms and an analysis of a very large dataset.

The depth of the applicability of this project will depend upon survey results to evaluate how people might want to take advantage of such a tool in contexts beyond research. The program could potentially identify tweets on Twitter as they are made, and if the tweet is revealing a personal experience, have utility in that it could inform the appropriate organization (police, school, women’s shelter, etc.) of the assault/harassment, connect the victim with another victim to talk, or even just archive the Tweet so that evidence of it exists even after it has been deleted.
