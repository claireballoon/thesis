\section{Problem}

\begin{quote}
``If all women who have been sexually harassed or assaulted wrote `Me too.' as a status, we might give people a sense of the magnitude of the problem." - Alyssa Milano
\end{quote}

Beginning in October 2017, victims of sexual assault or harassment began using the hashtag “\#MeToo” to illustrate the magnitude and prevalence of sexual assault and sexual harassment. With the growth of the movement, the hashtag \#MeToo is used widely on social media from people who are in support of the movement but not victims themselves, from people who are antagonists or critics of the movement, in general discussion and news coverage as well as for its original purpose of victims communicating a personal experience. As the study of sexual assault and harassment grows in relevance and popularity, the \#MeToo movement exists as an unprecedented platform to be vocal about personal experiences regarding sexual assault and harassment. This thesis explores this platform and endeavors to use it to draw new conclusions about the demographic groups who experience these problems.

\section{Scope}

This thesis only covers a basic classification implementation and also provides an online interface to allow others to use the classification tool without compiling and processing on their local machine. However, research indicates many other possible avenues through which this program might be applied. These possibilities and applications are discussed within the Supporting Materials chapter.

\section{Overview}

\subsection{Natural Language Processing (NLP)}

blah

\subsection{Supervised Machine Learning}

blah

\subsection{Classification Model}

blah
